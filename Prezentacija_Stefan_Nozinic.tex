\documentclass{beamer}
\usetheme{Madrid}

\usepackage[utf8]{inputenc}
\usepackage[T1]{fontenc}
\usepackage[serbian]{babel}

\title[Kritika: AI-Based Fault-Proneness Metrics]{Kritika na rad: AI-Based Fault-Proneness Metrics for Source Code}
\institute{Univerzitet u Novom Sadu}
\author{Stefan Nožinić}
\date{\today}

\begin{document}

\begin{frame}
  \titlepage
\end{frame}

\begin{frame}
\frametitle{Uvod}
\begin{itemize}
    \item Pregled problema u razvoju softvera.
    \item Važnost \textit{Code Review} procesa.
    \item Izazovi u detekciji grešaka u izvornom kodu.
    \item Kratka analiza pristupa koji rad predstavlja.
\end{itemize}
\end{frame}

\begin{frame}
\frametitle{Ulazni podaci}
\begin{itemize}
    \item Detalji o ulaznim podacima rada.
    \item Izvor podataka i njihova obrada.
    \item Kritički osvrt na kvalitet i relevantnost podataka.
\end{itemize}
\end{frame}

\begin{frame}
\frametitle{Metode}
\begin{itemize}
    \item Opis metodologije rada.
    \item Analiza metrika zasnovanih na stablima i vektorskom prostoru.
    \item Kritika primenjenih metoda i njihova efikasnost.
    \item Prednosti i mane navedenih metoda.
\end{itemize}
\end{frame}

\begin{frame}
\frametitle{Unapređenja}
\begin{itemize}
    \item Predlozi za poboljšanja u budućim istraživanjima.
    \item Ideje za implementaciju metoda u stvarnim projektima.
    \item Smerovi za proširenje istraživanja.
\end{itemize}
\end{frame}

\begin{frame}
\frametitle{Zaključak}
\begin{itemize}
    \item Konačna ocena rada.
    \item Potencijalni doprinosi rada u oblasti.
    \item Preporuke za dalje istraživanje.
\end{itemize}
\end{frame}

\end{document}
