\documentclass{beamer}
\usetheme{Madrid}

\usepackage[utf8]{inputenc}
\usepackage[T1]{fontenc}
\usepackage[serbian]{babel}

% do not include footer 
\setbeamertemplate{footline}{}

\title[Kritika: AI-Based Fault-Proneness Metrics]{Kritika na rad: AI-Based Fault-Proneness Metrics for Source Code}
\institute{Univerzitet u Novom Sadu}
\author{Stefan Nožinić}
\date{\today}

\begin{document}

\begin{frame}
  \titlepage
\end{frame}

\begin{frame}
\frametitle{Uvod}
\begin{itemize}
    \item Pregled problema u razvoju softvera.
    \item Važnost \textit{Code Review} procesa.
    \item Izazovi u detekciji grešaka u izvornom kodu.
    \item Kratka analiza pristupa koji rad predstavlja.
\end{itemize}
\end{frame}

\begin{frame}
\frametitle{Ulazni podaci}
\begin{itemize}
    \item Detalji o ulaznim podacima rada.
    \item Izvor podataka i njihova obrada.
    \begin{itemize}
        \item Open source projekti 
        \item Java programski jezik
        \item Promene nad metodama
    \end{itemize}
\end{itemize}
\end{frame}

\begin{frame}
\frametitle{Metode}
\begin{itemize}
    \item Opis metodologije rada.
    \item Analiza metrika zasnovanih na stablima 
    \item Analiza metrika zasnovanih na transformacijama u vektor
\end{itemize}
\end{frame}

\begin{frame}
    \frametitle{Poređenje metoda}
    \begin{table}[ht]
    \centering
    \begin{tabular}{|c|c|c|}
    \hline
    Metoda & Koeficijent koeralcije & p-vrednost \\
    \hline
    STK        & 0.6 &0.0000624 \\
    CodeBERT   & 0.52&0.0000169 \\
    PTK        & 0.40    & 0.00000518         \\
    SSTK   &   0.31  & 0.000443         \\
    \hline
    \end{tabular}
    \end{table}
\end{frame}
    


\begin{frame}
\frametitle{Unapređenja}
\begin{itemize}
    \item Ideje za implementaciju metoda u stvarnim projektima.
    \item Smerovi za proširenje istraživanja.
    \begin{itemize}
        \item Različiti programski jezici
        \item Evaluacija na specifičnoj kategoriji promena
        \begin{itemize}
            \item Po tipu promene
            \item Po broju linija koda
            \item Po broju metoda
            \item Po broju klasa
            \item Po broju fajlova
        \end{itemize}
        \item Različiti tipovi promena
        \begin{itemize}
            \item Dodavanje novih metoda
            \item Brisanje metoda
        \end{itemize}
        \item Upotreba drugih metoda kao npr GraphCodeBERT i sl.
    \end{itemize}
\end{itemize}
\end{frame}

\begin{frame}
\frametitle{Zaključak}
\begin{itemize}
    \item Konačna ocena rada.
    \item Potencijalni doprinosi rada u oblasti.
    \item Preporuke za dalje istraživanje.
\end{itemize}
\end{frame}

\end{document}
