\documentclass{article}
\usepackage{amsmath}
\usepackage{amssymb}
\usepackage{graphicx}
\usepackage{subfig}
\usepackage[serbian]{babel}
\usepackage{pgffor}
\begin{document}
\title{Izveštaj}
\author{Stefan Nožinić}
\maketitle
\section{Uvod}

U ovom izveštaju je analiziran odnos između različitih slučajnih promenljivih. 

\section{Metodologija}

U ovom radu je korišćen Spermansov koeficijent korelacije za određivanje zavisnosti između dve slučajne promenljive. Spermansov koeficijent korelacije je definisan kao:

\begin{equation}
\rho = 1 - \frac{6\sum_{i=1}^{n}d_{i}^{2}}{n(n^{2}-1)}
\end{equation}

gde je $d_{i}$ rastojanje između rangova dve slučajne promenljive. Rangovi se određuju tako da se sortiraju vrednosti slučajnih promenljivih i dodeljuju im se rangovi od 1 do n. Ako se dve vrednosti poklapaju, dodeljuje im se aritmetička sredina njihovih rangova.

Razlog zašto je korišćen Spermansov koeficijent korelacije je taj što je on nezavisan od raspodele slučajnih promenljivih. 
U ovom slučaju raspodela promenljivih nije bila normalna, pa je korišćen Spermansov koeficijent korelacije.

Na narednim slikama su prikazane raspodele slučajnih promenljivih.


% loop over variables
\foreach \x in {Cine, Fine Arts, Halls, Heritage, Literat, Music, Sport, Theater} {
\begin{figure}[h!]
\centering
\includegraphics{figures/\x.png}
\caption{Raspodela slučajne promenljive \x}
\end{figure}
}

\section{Rezultati}

Na sledećim slikama su prikazani korelacioni koeficijenti za različite slučajne promenljive.

\foreach \x in {1,2,3,None} {
\begin{figure}
\centering
\subfloat[Korelacije za region \x]{\includegraphics[width=0.5\textwidth]{figures/Correlogram_\x.png}}
\end{figure}
}

\section{Zaključak}

\end{document}
\endinput