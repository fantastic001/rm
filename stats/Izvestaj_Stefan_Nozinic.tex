\documentclass{article}
\usepackage{amsmath}
\usepackage{amssymb}
\usepackage{graphicx}
\usepackage{subfig}
\usepackage[serbian]{babel}
\usepackage[utf8]{inputenc}
\usepackage{fontenc}
\usepackage{pgffor}
\begin{document}
\title{Izveštaj}
\author{Stefan Nožinić}
\maketitle
\section{Uvod}

U ovom izveštaju je analiziran budžet za kulturu u različitim gradovima u Portugalu. 

\section{Podaci}

Podaci sadrže gradove gde svaki grad ima region kom pripada koji je jedan od:

\begin{itemize}
    \item Altenejo
    \item Centralni deo 
    \item Severni deo 
\end{itemize}

Pored regiona, podaci sadrže i procenat budžeta za kuluturu na različite kategorije odnosno:

\begin{itemize}
    \item Bioskop
    \item Umetnost
    \item Sale za događaje
    \item Nasleđe
    \item Književnost
    \item Muzika
    \item Sport
    \item Pozorište
\end{itemize}

\section{Metodologija}

U ovom radu je korišćen Spermanov koeficijent korelacije za određivanje zavisnosti između dve slučajne promenljive. Spermanov koeficijent korelacije je definisan kao:

\begin{equation}
\rho = 1 - \frac{6\sum_{i=1}^{n}d_{i}^{2}}{n(n^{2}-1)}
\end{equation}


gde je $d_{i}$ rastojanje između rangova dve slučajne promenljive. Rangovi se određuju tako da se sortiraju vrednosti slučajnih promenljivih i dodeljuju im se rangovi od 1 do n. Ako se dve vrednosti poklapaju, dodeljuje im se aritmetička sredina njihovih rangova.

Razlog zašto je korišćen Spermanov koeficijent korelacije je taj što je on nezavisan od raspodele slučajnih promenljivih. 
U ovom slučaju raspodela promenljivih nije bila normalna što se može i videti na slikama \ref{fig:Bioskop}-\ref{fig:Pozorište}.



% loop over variables
\foreach \x in {Bioskop, Umetnost, Sale za događaje, Nasleđe, Književnost, Muzika, Sport, Pozorište} {
\begin{figure}[h!]
\centering
\includegraphics{figures/\x.png}
\caption{Raspodela za kategoriju: \x}
\label{fig:\x}
\end{figure}
}

\section{Rezultati}

Na slikama \ref{fig:Correlogram_Altenejo}-\ref{fig:Correlogram_svi}  su prikazani korelacioni koeficijenti za različite regione.

\foreach \x in {Altenejo,Centralni deo,Severni deo,svi} {
\begin{figure}
\centering
\includegraphics{figures/Correlogram_\x.png}
\caption{Korelacije za region \x}
\label{fig:Correlogram_\x}
\end{figure}
}

\subsection{Korelacije za region Altenejo}


Ono što se može zaključiti iz ovih slika je da postoji pozitivna korelacija između budžeta za pozorište i budžeta za umetnost koja iznosi 0.61. 

Takođe, postoji negativna korelacija između budžeta za sale za događaje i budžeta za Književnost koja iznosi -0.51.

Negativna korelacija je uočena i kod budžeta za Muziku i budžeta za sport koja iznosi -0.53.

\subsection{Korelacije za region Centralni deo}

Ono što se može zaključiti iz ovih slika je da postoji pozitivna korelacija između budžeta za pozorište i budžeta za bioskop koja iznosi 0.53.

Takođe, postoji negativna korelacija između budžeta za sport i budžeta za muziku koja iznosi -0.56.

\subsection{Korelacije za region Severni deo}

U severnom delu zemlje nije uočena nikakva pozitivna ni negativna korelacija između budžeta za različite kategorije.

\section{Diskusija}

Na osnovu dobijenih rezultata može se zaključiti da postoji pozitivna korelacija između budžeta za pozorište i budžeta za umetnost. Ovo je očekivano, jer se u pozorištu izvode različite umetničke predstave a u regionu Altenejo se vidi kordinisana raspodela budžeta za ove dve kategorije. 

Negativna korelacija između budžeta za muziku i budžeta za sport je takođe očekivana, jer gradovi svoje ograničene resurse pokušavaju da usmere skladno svojoj infrastrukturi. 

Kada se pogledaju negativne korelacije, može se zaključiti da gradovi mahom svoje budžete fokusiraju na određene kategorije i da im se fokus razlikuje, bez obzira što su geografski blizu i nalaze se u istim regionima. 


\section{Zaključak}

U ovom izveštaju je analiziran budžet za kulturu u različitim gradovima u Portugalu.

Korišćen je Spermansov koeficijent korelacije za određivanje zavisnosti između dve slučajne promenljive.

Na osnovu dobijenih rezultata može se zaključiti da postoji pozitivna korelacija između budžeta za pozorište i budžeta za umetnost u regionu Altenejo.

Zaključuje se da su kategorije koje su infrastrukturno kompatibilne često i kordinisane u dodeli budžeta. Kategorije kutlure koje su infrastrukturno inkompatibilne su često i negativno korelisane u dodeli budžeta kao na primer kategorije muzike i sporta.

\section{Literatura}

\begin{enumerate}
    \item \textit{INE - Instituto Nacional de Estatística, Portugal}
    \item \textit{https://en.wikipedia.org/wiki/Spearman\%27s\_rank\_correlation\_coefficient}
\end{enumerate}

\end{document}
\endinput